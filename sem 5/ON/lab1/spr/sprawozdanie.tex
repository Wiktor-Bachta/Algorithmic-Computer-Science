\documentclass[12pt, letterpaper]{article}
\usepackage[utf8]{inputenc}
\usepackage[T1]{fontenc}
\title{Obliczenia Naukowe - sprawozdanie nr 1}
\author{Wiktor Bachta}
\date{Październik 2024}

\begin{document}

\maketitle

\section{Zadanie 1}

\subsection{Opis}

Wyznaczanie iteracyjne liczb \textit{macheps, eta, max}
dla różnych typów arytmetyki zmiennoprzecinkowej.

\subsection{Rozwiązanie}

Liczby \textit{eta} oraz \textit{macheps} wyznaczam przez iteracyjne dzielenie
przez 2.
Liczbę \textit{max} wyznaczam przez mnożenie przez 2, a następnie powrót
funkcją \textit{prevfloat}

\subsection{Wynik}

\begin{center}
    \begin{tabular}{|c | c | c| c |}
        \hline
        typ     & \textit{macheps}      & \textit{eps(typ)}     & wartość z
        float.h
        \\
        \hline
        Float16 & 0.000977              & 0.000977              & -
        \\
        \hline
        Float32 & 1.1920929e-7          & 1.1920929e-7          & 1.192093e-07
        \\
        \hline
        Float64 & 2.220446049250313e-16 & 2.220446049250313e-16 &
        2.220446e-16
        \\
        \hline
    \end{tabular}
\end{center}

\begin{center}
    \begin{tabular}{|c | c | c|}
        \hline
        typ     & \textit{eta} & \textit{nextfloat(typ(0))} \\
        \hline
        Float16 & 6.0e-8       & 6.0e-8                     \\
        \hline
        Float32 & 1.0e-45      & 1.0e-45                    \\
        \hline
        Float64 & 5.0e-324     & 5.0e-324                   \\
        \hline
    \end{tabular}
\end{center}

\begin{center}
    \begin{tabular}{|c | c | c| c |}
        \hline
        typ     & \textit{max}           & \textit{floatmax(typ)} & wartość z
        float.h
        \\
        \hline
        Float16 & 6.55e4                 & 6.55e4                 & -
        \\
        \hline
        Float32 & 3.4028235e38           & 3.4028235e38           &
        3.402823e+38
        \\
        \hline
        Float64 & 1.7976931348623157e308 & 1.7976931348623157e308 &
        1.797693e+308
        \\
        \hline
    \end{tabular}
\end{center}

\begin{center}
    \begin{tabular}{|c | c | c|}
        \hline
        typ     & \textit{floatmin(typ)}  & \textit{$Min_{nor}$}
        \\
        \hline
        Float32 & 1.1754944e-38           & 1.1754944e-38
        \\
        \hline
        Float64 & 2.2250738585072014e-308 & 2.2250738585072014e-308
        \\
        \hline
    \end{tabular}
\end{center}

\subsection{Wnioski}

\begin{itemize}
    \item Iteracyjne obliczenia dają prawidłowe wartości dla \textit{macheps,
              eta, max}
    \item Jaki związek ma liczba \textit{macheps} z precyzją arytmetyki?
          $\textit{macheps} = 2 \epsilon$ - maksymalna odległość od
          reprezentowalnej
          liczby jest w środku dwóch kolejnych liczb
    \item Jaki związek ma liczba eta z liczbą $MIN_{sub}$?
          Są tożsame - najmniejsza liczba zdenormalizowana.
    \item Co zwracają funkcje $floatmin(Float32)$ i $floatmin(Float64)$ i jaki
          jest związek zwracanych wartości z liczbą $MIN_{nor}$?
          Są tożsame - najmniejsza liczba znormalizowana.
\end{itemize}

\clearpage

\section{Zadanie 2}

\subsection{Opis}

Wyznaczenie epsilonu maszynowego \textit{macheps} na podstawie wzoru
\[ macheps = 3*(4/3 - 1) - 1 \]

\subsection{Wynik}

\begin{center}
    \begin{tabular}{|c | c | c|}
        \hline
        Typ     & \textit{macheps}      & \textit{macheps} wg wzoru Kahana \\
        \hline
        Float16 & 0.000977              & -0.000977                        \\
        \hline
        Float32 & 1.1920929e-7          & 1.1920929e-7                     \\
        \hline
        Float64 & 2.220446049250313e-16 & -2.220446049250313e-16           \\
        \hline
    \end{tabular}
\end{center}

\subsection{Wnioski}

Wzór Kahana może być wykorzystany do obliczenia \textit{macheps}, przy
zastosowaniu wartości bezwzględnej.
Różnice znaków wynikają z miejsca "ucięcia" liczby 4/3 w odpowiedniej
reprezentacji.
Dla Float16 oraz Float64 jest ona zaokrąglana w dół (...01), natomiast dla
Float32 w górę (...11).

\section{Zadanie 3}

\subsection{Opis}

Sprawdzenie rozmieszczenia liczb typu Float64 w przedziale $[1, 2]$.
Każda liczba $x$ w tym przedziale może być przedstawiona jako $x + \delta k$,
gdzie
$ k = 1, 2,...,2^{52} - 1$ i $\delta = 2^{-52}$
Podobne rozważania dla przedziałów $[1/2, 1]$ oraz $[2, 4]$

\subsection{Rozwiązanie}

Przenalalizuję reprezentację bitową liczb w przedziale, z najmniejszym możliwym
krokiem
$\delta = nextfloat(x)$

\subsection{Wynik}

\begin{table}[h]
    \centering
    \begin{tabular}{|c | c|}
        \hline
        liczba           & \textit{bitstring} (liczba)                   \\
        \hline
        1.0              &
        0011111111110000000000000000000000000000000000000000000000000000 \\
        \hline
        1.0 + $\delta$   &
        0011111111110000000000000000000000000000000000000000000000000001 \\
        \hline
        1.0 + 2 $\delta$ &
        0011111111110000000000000000000000000000000000000000000000000010 \\
        \hline
        1.0 + 3 $\delta$ &
        0011111111110000000000000000000000000000000000000000000000000011 \\
        \hline
        1.0 + 4 $\delta$ &
        0011111111110000000000000000000000000000000000000000000000000100 \\
        \hline
    \end{tabular}
    \caption{$\delta = 2^{-52}$}
\end{table}

\begin{table}[h]
    \centering
    \begin{tabular}{|c | c|}
        \hline
        liczba           & \textit{bitstring} (liczba)                   \\
        \hline
        0.5              &
        0011111111100000000000000000000000000000000000000000000000000000 \\
        \hline
        0.5 + $\delta$   &
        0011111111100000000000000000000000000000000000000000000000000001 \\
        \hline
        0.5 + 2 $\delta$ &
        0011111111100000000000000000000000000000000000000000000000000010 \\
        \hline
        0.5 + 3 $\delta$ &
        0011111111100000000000000000000000000000000000000000000000000011 \\
        \hline
        0.5 + 4 $\delta$ &
        0011111111100000000000000000000000000000000000000000000000000100 \\
        \hline
    \end{tabular}
    \caption{$\delta = 2^{-53}$}
\end{table}

\begin{table}[h]
    \centering
    \begin{tabular}{|c | c|}
        \hline
        liczba           & \textit{bitstring} (liczba)                   \\
        \hline
        2.0              &
        0100000000000000000000000000000000000000000000000000000000000000 \\
        \hline
        2.0 + $\delta$   &
        0100000000000000000000000000000000000000000000000000000000000001 \\
        \hline
        2.0 + 2 $\delta$ &
        0100000000000000000000000000000000000000000000000000000000000010 \\
        \hline
        2.0 + 3 $\delta$ &
        0100000000000000000000000000000000000000000000000000000000000011 \\
        \hline
        2.0 + 4 $\delta$ &
        0100000000000000000000000000000000000000000000000000000000000100 \\
        \hline
    \end{tabular}
    \caption{$\delta = 2^{-51}$}
\end{table}

\clearpage

\subsection{Wnioski}

Ze względu na to, że dane przedziały są w postaci $[2^{t}, 2^{t + 1}]$ cecha
dla każdej liczby z przedziału
będzie taka sama. Czyli liczba jest tożsama z mantysą, którą możemy ułożyć na
$2^{52}$ sposobów.
Najmniejszy krok jest wtedy równy $2^{-(52 + t)}$ i liczby są ułożone w
przedziale jednostajnie. Oznacza to, że precyzja reprezentacji jest
proporcjonalna do odległości liczby od zera - dokładniejsza dla mniejszych liczb.

\section{Zadanie 4}

\subsection{Opis}

Znaleźć najmniejszą liczbę w arytmetyce Float64 $ 1 < x < 2$, dla której
$x(1/x) \neq 1$

\subsection{Rozwiązanie}

Iteracyjne przechodzenie po liczbach za pomocą \textit{nextfloat} do
odnalezienia liczby o szukanej własności

\subsection{Wynik}

Znaleziona liczba: 1.000000057228997

$x(1/x) = 0.9999999999999999$

\subsection{Wnioski}

Liczba $1/x$ nie ma dokładnej reprezentacji
w arytmetyce Float, więc ulega zaokrągleniu do najbliższej możliwej liczby, co
powoduje błąd. Nawet najprostrze dzielenie i monożenie wprowadza potencjalne
błędy, a komputer nie potrafi "od tak" upraszczać wyrażeń.

\section{Zadanie 5}

\subsection{Opis}

Obliczanie iloczynu skalarnego dwóch wektorów za pomocą czterech metod

\[ x = [2.718281828, -3.141592654, 1.414213562, 0.5772156649, 0.3010299957] \]
\[ y = [1486.2497, 878366.9879, -22.37492, 4773714.647, 0.000185049] \]

\begin{itemize}
    \item Obliczanie i sumowanie iloczynów po kolei
    \item Obliczanie i sumowanie iloczynów w odwrotnej kolejności
    \item Od największego do najmniejszego (dodaj dodatnie liczby w porządku od
          największego
          do najmniejszego, dodaj ujemne liczby w porządku od najmniejszego do
          największego,
          a następnie daj do siebie obliczone sumy częściowe)
    \item od najmniejszego do największego
\end{itemize}

\subsection{Wynik}

\begin{center}
    \begin{tabular}{|c | c | c | c | c |}
        \hline
        Typ     & metoda 1               & metoda 2                & metoda 3 &
        metoda 4
        \\
        \hline
        Float32 & -0.4999443             & -0.4543457              & -0.5     &
        -0.5
        \\
        \hline
        Float64 & 1.0251881368296672e-10 & -1.5643308870494366e-10 & 0.0      &
        0.0
        \\
        \hline
    \end{tabular}
\end{center}

\subsection{Wnioski}

Rzeczywista wartość iloczynu wynosi $-1.00657107000000 10^{-11}$
Obie arytmetyki dały wyniki dalekie rzeczywitej wartości dla każdej z metod.
Wszystkie błędy względne przekraczają tutaj $100\%$.
Oznacza to, że nawet w podstawowych operacjach (dodawanie, mnożenie) kolejność
wykonywanych działań ma znaczący wpływ na otrzymywane rezultaty. Należy stosować
odpowiednie metody/algorytmy w zależności o wykonywanych obliczeń.

\section{Zadanie 6}

\subsection{Opis}

Obliczenie i porówananie wartości dwóch funkcji, które są sobie równe, ale
wykorzystują inne wzory

\[ f(x) = \sqrt{x^2 + 1} - 1 \]
\[ g(x) = x^2 / {(\sqrt{x^2 + 1} + 1)} \]

dla kolejnych argumentów $x = 8^{-1}, 8^{-2},...$

\subsection{Wynik}

\begin{center}
    \begin{tabular}{|c | c | c | c |}
        \hline
        $x$        & $f(x)$                 & $g(x)$                 & $x^2 /
            2$
        \\
        \hline
        $8^{-1}$   & 0.0077822185373186414  & 0.0077822185373187065  &
        0.0078125
        \\
        \hline
        $8^{-2}$   & 0.00012206286282867573 & 0.00012206286282875901 &
        0.0001220703125
        \\
        \hline
        $8^{-3}$   & 1.9073468138230965e-6  & 1.907346813826566e-6   &
        1.9073486328125e-6
        \\
        \hline
        $8^{-4}$   & 2.9802321943606103e-8  & 2.9802321943606116e-8  &
        2.9802322387695312e-8
        \\
        \hline
        $8^{-5}$   & 4.656612873077393e-10  & 4.6566128719931904e-10 &
        4.656612873077393e-10
        \\
        \hline
        $8^{-6}$   & 7.275957614183426e-12  & 7.275957614156956e-12  &
        7.275957614183426e-12
        \\
        \hline
        $8^{-7}$   & 1.1368683772161603e-13 & 1.1368683772160957e-13 &
        1.1368683772161603e-13
        \\
        \hline
        $8^{-8}$   & 1.7763568394002505e-15 & 1.7763568394002489e-15 &
        1.7763568394002505e-15
        \\
        \hline
        $8^{-9}$   & 0.0                    & 2.7755575615628914e-17 &
        2.7755575615628914e-17
        \\
        \hline
        $8^{-10}$  & 0.0                    & 4.336808689942018e-19  &
        4.336808689942018e-19
        \\
        \hline
        \dots      & \dots                  & \dots                  & \dots
        \\
        \hline
        $8^{-177}$ & 0.0                    & 1.012e-320             &
        1.012e-320
        \\
        \hline
        $8^{-178}$ & 0.0                    & 1.6e-322               & 1.6e-322
        \\
        \hline
        $8^{-179}$ & 0.0                    & 0.0                    & 0.0
        \\
        \hline

    \end{tabular}
\end{center}

\subsection{Wnioski}

Analiza liczby $\sqrt{x^2 + 1}$ pozwala na dokładniejsze zrozumienie obliczeń.
Jako, że $x$ jest bliski zeru, $x^2$ staje się bardzo mała i dodawanie z jedynką
daje rezultat równy jeden. Cecha $(8^{-9})^2 = 2^{-54}$ to -54, cecha jedynki to 0,
czyli ich różnica przekracza długość mantysy.
Dlatego funkcja $f$ daje wartości równe 0, a $g$ w rzeczywistości zwraca $x^2 /
    2$
Z tego powodu, wiarygodniejsza jest funkcja $g$. 
Wnioskiem jest fakt, że dla dużych różnic wielkości liczb $x + y = x$, na co należy uważać przy obliczeniach.
Drugi wniosek to fakt, że dla pewnych obliczeń można znaleźć alternatywny wzór, który działa z większą dokładnością. 

\section{Zadanie 7}

\subsection{Opis}

Przybliżanie pochodnej w punkcie za pomocą dedfinicji, dla coraz mniejszych $h$
i porównanie z wartością rzeczywistą w punkcie $x = 1$.

\[ f(x) = sin(x) + cos(3 x) \]
\[ f'(x) = cos(x) - 3 sin(3 x) \]
\[ \widetilde{f} (x) = \frac{f(x + h) - f(x)}{h} \]
\[ h = 2^{-n}, n = 0, 1, 2,...,54 \]

\subsection{Wynik}

\begin{center}
    \begin{tabular}{|c | c | c | c |}
        \hline
        $h$                 & $h + 1$             &
        $\widetilde{f} (x)$ &
        $|f'(x) -
            \widetilde{f} (x)|$
        \\
        \hline
        $2^{0}$             & 2.0                 &
        2.0179892252685967  &
        1.9010469435800585
        \\
        \hline
        $2^{-1}$            & 1.5                 &
        1.8704413979316472  &
        1.753499116243109
        \\
        \hline
        $2^{-2}$            & 1.25                &
        1.1077870952342974  &
        0.9908448135457593
        \\
        \hline
        \dots               & \dots               & \dots
                            &
        \dots
        \\
        \hline
        $2^{-26}$           & 1.0000000149011612
                            & 0.11694233864545822 &
        5.6956920069239914e-8
        \\
        \hline
        $2^{-27}$           & 1.0000000074505806
                            & 0.11694231629371643 &
        3.460517827846843e-8
        \\
        \hline
        $2^{-28}$           & 1.0000000037252903
                            & 0.11694228649139404 &
        4.802855890773117e-9
        \\
        \hline
        \dots               & \dots               & \dots
                            &
        \dots
        \\
        \hline
        $2^{-52}$           & 1.0000000000000002
                            & -0.5                &
        0.6169422816885382
        \\
        \hline
        $2^{-53}$           & 1.0                 & 0.0
                            &
        0.11694228168853815
        \\
        \hline
        $2^{-54}$           & 1.0                 & 0.0
                            &
        0.11694228168853815
        \\
        \hline
    \end{tabular}
\end{center}

\subsection{Wnioski}

Najlepsze przybliżenie otrzymujemy dla $n=28$, potem wartość błędu rośnie.
Dla $h <= 2^{-53}, h + 1 = 1$ co psuje przybliżenie pochodnej, dając wynik 0.
Jak wytłumaczyć, że od pewnego momentu zmniejszanie wartości h nie poprawia
przybliżenia wartości pochodnej?
Błędy zaokrągleń: Gdy
h jest zbyt małe, wyrażenie $f(x + h) - f(x)$ staje się podatne na błędy
zaokrągleń. Różnica
jest liczona na liczbach o małej różnicy wartości, co prowadzi do coraz
większego wpływu błędów numerycznych.
Wniosek: obliczenia maszynowne rzadko zachowują się intuicyjnie, i dosłowne
traktowanie matematycznych granic, może dawać gorsze rezultaty dla teoretycznie
lepszych (dokładniejszych) wartości. Odejmownie liczb barzo bliskich powoduje dużą
utratę dokładności.
\end{document}
