\documentclass[12pt, letterpaper]{article}
\usepackage[utf8]{inputenc}
\usepackage[T1]{fontenc}
\usepackage{amsmath}
\usepackage{caption}
\title{AOD - sprawozdanie nr 2}
\author{Wiktor Bachta}
\date{Listopad 2024}

\begin{document}

\maketitle

\section{Wstęp}

Wszystkie zadania rozwiązałem w języku Julia z wykorzystaniem biblioteki JuMP
oraz GLPK.

\section{Zadanie 1}

Model ten rozwiązuje problem minimalizacji kosztów zakupu paliwa dla
przedsiębiorstwa lotniczego, które musi dostarczyć określoną ilość paliwa na
lotniska od różnych dostawców.

\subsection{Parametry}
\begin{itemize}
  \item \( m \): liczba dostawców,
  \item \( n \): liczba lotnisk,
  \item \( d_j \): zapotrzebowanie na paliwo na lotnisku \( j \),
  \item \( p_i \): maksymalna ilość paliwa dostępna od dostawcy \( i \),
  \item \( c_{ij} \): koszt dostarczenia paliwa od dostawcy \( i \) na lotnisko
        \( j \).
\end{itemize}

\subsection{Zmienne decyzyjne}
Definiujemy zmienne decyzyjne \( x_{ij} \), które reprezentują ilość paliwa
dostarczanego przez firmę \( i \) na lotnisko \( j \), gdzie:
\[
  x_{ij} \geq 0
\]

\subsection{Funkcja celu}
Celem jest minimalizacja całkowitego kosztu dostarczenia paliwa. Koszt
dostarczenia jednego galonu paliwa przez dostawcę \( i \) na lotnisko \( j \)
jest zapisany w macierzy kosztów \( c_{ij} \). Funkcja celu wyraża się
następująco:
\[
  \min \sum_{i=1}^m \sum_{j=1}^n c_{ij} x_{ij}
\]

\subsection{Ograniczenia}
\begin{enumerate}
  \item \textbf{Ograniczenia zapotrzebowania na paliwo na każdym lotnisku}:
        Każde lotnisko \( j \) wymaga określonej ilości paliwa, oznaczonej jako
        \( d_j
        \). Dla każdego lotniska \( j \) formułujemy ograniczenie:
        \[
          \sum_{i=1}^m x_{ij} = d_j
        \]

  \item \textbf{Ograniczenia produkcyjne dostawców}: Każdy dostawca \( i \)
        ma ograniczoną ilość paliwa, którą może dostarczyć (zdefiniowaną jako
        \( p_i
        \)). Sformułowanie tych ograniczeń dla każdego dostawcy \( i \) wygląda
        następująco:
        \[
          \sum_{j=1}^n x_{ij} \leq p_i
        \]
\end{enumerate}

\subsection{Egzemplarz}
\begin{center}
  \begin{tabular}{|c|c|c|c|}
    \hline
              & Firma 1 & Firma 2 & Firma 3 \\
    \hline
    \( p_j \) & 275000  & 550000  & 660000  \\
    \hline
  \end{tabular}
\end{center}

\begin{center}
  \begin{tabular}{|c|c|c|c|c|}
    \hline
              & Lotnisko 1 & Lotnisko 2 & Lotnisko 3 & Lotnisko 4 \\
    \hline
    \( d_i \) & 110000     & 220000     & 330000     & 440000     \\
    \hline
  \end{tabular}
\end{center}

\begin{center}
  \begin{tabular}{|c|c|c|c|}
    \hline
    \( c_{ij} \) & Firma 1 & Firma 2 & Firma 3 \\
    \hline
    Lotnisko 1   & 10      & 7       & 8       \\
    Lotnisko 2   & 10      & 11      & 14      \\
    Lotnisko 3   & 9       & 12      & 4       \\
    Lotnisko 4   & 11      & 13      & 9       \\
    \hline
  \end{tabular}
\end{center}

\subsection{Rozwiązanie}
\begin{center}
  \begin{tabular}{|c|c|c|c|}
    \hline
    \( x_{ij} \) & Firma 1 & Firma 2 & Firma 3 \\
    \hline
    Lotnisko 1   & 0       & 110000  & 0       \\
    Lotnisko 2   & 165000  & 55000   & 0       \\
    Lotnisko 3   & 0       & 0       & 330000  \\
    Lotnisko 4   & 110000  & 0       & 330000  \\
    \hline
  \end{tabular}
\end{center}

\begin{itemize}
  \item Jaki jest minimalny łączny koszt dostaw wymaganych ilosci paliwa na
        wszystkie lotniska? \\ 8525000\$
  \item Czy wszystkie firmy dostarczają paliwo? \\ Tak, wszystkie 3 firmy
        dostarczają paliwo.
  \item Czy możliwości dostaw paliwa przez firmy są wyczerpane? \\ Tak - dla
        firmy 1 i 3.
\end{itemize}

\section{Zadanie 2}

Model ten rozwiązuje problem optymalizacji produkcji zakładu, który może
wytwarzać cztery różne wyroby w ograniczonych ilościach, uwzględniając czas
przetwarzania na trzech różnych maszynach oraz popyt rynkowy. Celem modelu jest
maksymalizacja zysku.

\subsection{Parametry}
\begin{itemize}
  \item \( n \): liczba wyrobów,
  \item \( m \): liczba maszyn,
  \item \( p_i \): cena sprzedaży wyrobu \( P_i \),
  \item \( m_i \): koszt materiałowy na kilogram wyrobu \( P_i \),
  \item \( k_j \): koszt pracy maszyny \( M_j \) za minutę,
  \item \( t_{ij} \): czas przetwarzania wyrobu \( P_i \) na maszynie \( M_j
        \),
  \item \( c_j \): dostępny czas pracy dla maszyny \( M_j \),
  \item \( d_i \): maksymalny popyt na wyrób \( P_i \).
\end{itemize}

\subsection{Zmienna decyzyjna}

\( x_i \): liczba kilogramów produktu \( P_i \), które
należy
wyprodukować.

\subsection{Funkcja celu}
Celem jest maksymalizacja zysku, który można wyrazić jako różnicę między
przychodem ze sprzedaży wyrobów a kosztami zmiennymi, które obejmują koszty
materiałów oraz koszty pracy maszyn. Funkcja celu przyjmuje postać:
\[
  \max \left( \sum_{i=1}^n \left( p_i - m_i \right) x_i - \sum_{j=1}^m k_j
  \cdot \sum_{i=1}^n t_{ij} \cdot x_i \right)
\]

\subsection{Ograniczenia}

\begin{enumerate}
  \item \textbf{Ograniczenia dostępności czasu pracy maszyn}: Każda maszyna \(
        M_j \) ma ograniczony tygodniowy czas pracy, oznaczony jako \( c_j \),
        co
        formułujemy jako:
        \[
          \sum_{i=1}^n t_{ij} \cdot x_i \leq c_j, \quad j = 1, \dots, m
        \]

  \item \textbf{Ograniczenia popytu rynkowego}: Każdy produkt \( P_i \) posiada
        maksymalny popyt rynkowy oznaczony jako \( d_i \), co prowadzi do
        ograniczenia:
        \[
          x_i \leq d_i, \quad i = 1, \dots, n
        \]

\end{enumerate}

\subsection{Egzemplarz}

\begin{center}
  \begin{tabular}{|c|c|c|c|}
    \hline
    \( t_{ij} \) & Maszyna 1 & Maszyna 2 & Maszyna 3 \\
    \hline
    \( P_1 \)    & 5         & 10        & 6         \\
    \( P_2 \)    & 3         & 6         & 4         \\
    \( P_3 \)    & 4         & 5         & 3         \\
    \( P_4 \)    & 4         & 2         & 1         \\
    \hline
  \end{tabular}
\end{center}

\begin{center}
  \begin{tabular}{|c|c|c|}
    \hline
              & \( d_i \) & \( m_i \) \\
    \hline
    \( P_1 \) & 400       & 4         \\
    \( P_2 \) & 100       & 1         \\
    \( P_3 \) & 150       & 1         \\
    \( P_4 \) & 500       & 1         \\
    \hline
  \end{tabular}
\end{center}

\begin{center}
  \begin{tabular}{|c|c|c|}
    \hline
              & \( k_j \) & \( c_j \) \\
    \hline
    Maszyna 1 & 2         & 3600      \\
    Maszyna 2 & 2         & 3600      \\
    Maszyna 3 & 3         & 3600      \\
    \hline
  \end{tabular}
\end{center}

\subsection{Rozwiązanie}
\begin{center}
  \begin{tabular}{|c|c|}
    \hline
              & \( x_{i} \) \\
    \hline
    \( P_1 \) & 125         \\
    \( P_2 \) & 100         \\
    \( P_3 \) & 150         \\
    \( P_4 \) & 500         \\
    \hline
  \end{tabular}
\end{center}

Profit = 3632 $\frac{1}{2}$ \$

\section{Zadanie 3}

Model ten rozwiązuje problem optymalizacji produkcji i magazynowania towarów w
firmie, która posiada ograniczoną zdolność produkcyjną w trybie podstawowym
oraz dodatkowym, przy czym produkcja dodatkowa wiąże się z wyższym kosztem
jednostkowym. Firma musi spełnić zapotrzebowanie w każdym z okresów przy
minimalizacji kosztów produkcji i magazynowania.

\subsection{Parametry}
\begin{itemize}
  \item \( K \): liczba okresów,
  \item \( c_j \): koszt produkcji w trybie podstawowym w okresie \( j \),
  \item \( o_j \): koszt produkcji w trybie dodatkowym w okresie \( j \),
  \item \( h \): koszt magazynowania towaru przez jeden okres,
  \item \( b_j \): maksymalna produkcja podstawowa w okresie \( j \),
  \item \( a_j \): maksymalna produkcja dodatkowa w okresie \( j \),
  \item \( d_j \): zapotrzebowanie w okresie \( j \),
  \item \( s_0 \): początkowy stan magazynu,
  \item \( S \): maksymalna pojemność magazynu.
\end{itemize}

\subsection{Zmienne decyzyjne}
Wprowadzamy następujące zmienne decyzyjne:
\begin{itemize}
  \item \( x_j \): liczba jednostek wyprodukowanych w trybie podstawowym w
        okresie \( j \), gdzie \( x_j \geq 0 \),
  \item \( y_j \): liczba jednostek wyprodukowanych w trybie dodatkowym w
        okresie \( j \), gdzie \( y_j \geq 0 \),
  \item \( s_j \): liczba jednostek towaru przechowywanych w magazynie na
        koniec okresu \( j \), gdzie \( s_j \geq 0 \).
\end{itemize}

\subsection{Funkcja celu}
Celem jest minimalizacja łącznych kosztów produkcji i magazynowania towaru.
Funkcja celu przyjmuje postać:
\[
  \min \sum_{j=1}^K \left( c_j x_j + o_j y_j + h s_j \right)
\]

\subsection{Ograniczenia}
\begin{enumerate}
  \item \textbf{Ograniczenia zdolności produkcyjnych}:
        \[
          x_j \leq b_j
        \]
        \[
          y_j \leq a_j
        \]
        gdzie \( b_j \) i \( a_j \) to odpowiednio maksymalna liczba jednostek
        produkowanych w trybie podstawowym i dodatkowym w okresie \( j \).

  \item \textbf{Ograniczenia zapotrzebowania na towar}:
        Liczba jednostek towaru przechowywanych na koniec okresu \( j+1 \) musi
        zapewnić spełnienie zapotrzebowania w kolejnym okresie, zatem:
        \[
          s_{j+1} = s_j + x_j + y_j - d_j
        \]
        gdzie \( d_j \) to zapotrzebowanie w okresie \( j \).

  \item \textbf{Ograniczenia magazynowe}:
        Przechowywana liczba jednostek towaru na koniec każdego okresu nie może
        przekroczyć maksymalnej pojemności magazynu:
        \[
          s_j \leq S
        \]
        gdzie \( S \) to maksymalna liczba jednostek, które mogą być
        przechowywane.

  \item \textbf{Warunki początkowe i końcowe}:
        Na początku pierwszego okresu stan magazynu wynosi \( s_1 = s_0 \),
        gdzie \( s_0 \) to początkowa liczba jednostek towaru. Na koniec
        ostatniego
        okresu \( s_{K+1} = 0 \), co oznacza, że na koniec cyklu produkcyjnego
        magazyn
        ma zostać opróżniony.

  \item Niepotrzebne jest ograniczenie na wykorzystanie zasobu produktów
        bazowych przed rozpoczęciem dodatkowych,
        za względu na ich większy koszt produkcji.
\end{enumerate}

\subsection{Egzemplarz}

\begin{center}
  \begin{tabular}{|c|c|c|c|c|c|}
    \hline
    \( j \) & \( b_j \) & \( a_j \) & \( c_j \) & \( o_j \) & \( d_j \) \\
    \hline
    1       & 100       & 60        & 6000      & 8000      & 130       \\
    2       & 100       & 65        & 4000      & 6000      & 80        \\
    3       & 100       & 70        & 8000      & 10000     & 125       \\
    4       & 100       & 60        & 9000      & 11000     & 195       \\
    \hline
  \end{tabular}
\end{center}

\begin{center}
  \begin{tabular}{|c|c|}
    \hline
    \( h \)   & 1500 \\
    \( s_0 \) & 15   \\
    \( S \)   & 70   \\
    \hline
  \end{tabular}
\end{center}

\subsection{Rozwiązanie}

\begin{center}
  \begin{tabular}{|c|c|c|c|}
    \hline
    \( j \) & \( x_j \) & \( y_j \) & \( s_j \) \\
    \hline
    1       & 100       & 15        & 15        \\
    2       & 100       & 50        & 0         \\
    3       & 100       & 0         & 70        \\
    4       & 100       & 50        & 45        \\
    \hline
  \end{tabular}
\end{center}

\begin{itemize}
  \item Jaki jest minimalny łączny koszt produkcji i magazynowania towaru? \\
        3842500\$
  \item W których okresach firma musi zaplanowac produkcję ponadwymiarową? \\ W
        okresach 1,2 oraz 4
  \item W których okresach mozliwości magazynowania towaru są wyczerpane?  \\
        Przy nocy 2 -> 3 okres
\end{itemize}

\section{Zadanie 4}
Problem ten dotyczy znalezienia najtańszego połączenia między dwoma miastami w
sieci połączeń, gdzie całkowity czas przejazdu nie może przekroczyć zadanego
limitu.

\subsection{Parametry}
\begin{itemize}
  \item \( N \): zbiór miast,
  \item \( A \): zbiór połączeń między miastami, każdy w postaci \( (i, j,
        c_{ij}, t_{ij}) \), gdzie \( c_{ij} \) to koszt przejazdu, a \( t_{ij}
        \) to czas przejazdu,
  \item \( i^\circ \): miasto początkowe,
  \item \( j^\circ \): miasto docelowe,
  \item \( T \): maksymalny dopuszczalny czas przejazdu.
\end{itemize}

\subsection{Zmienne decyzyjne}
\begin{itemize}
  \item \( x_{ij} \): zmienna decyzyjna, która przyjmuje wartość \( 1 \),
        jeśli połączenie z miasta \( i \) do miasta \( j \) jest używane w
        optymalnej
        ścieżce, i \( 0 \) w przeciwnym przypadku.
\end{itemize}

\subsection{Funkcja celu}
Minimalizacja całkowitego kosztu przejazdu:
\[
  \min \sum_{(i, j) \in A} c_{ij} \cdot x_{ij}
\]

\subsection{Ograniczenia}
\begin{enumerate}
  \item \textbf{Ograniczenie dotyczące istniejących połaczeń}:
        \[
          x_{ij} = 0 \quad \text{jeżeli nie ma połączenia z i do j}
        \]
  \item \textbf{Ograniczenie czasu przejazdu}:
        \[
          \sum_{(i, j) \in A} t_{ij} \cdot x_{ij} \leq T
        \]
        gdzie \( T \) to maksymalny dopuszczalny czas przejazdu.

  \item \textbf{Warunek wyjścia z miasta początkowego \( i^\circ \)}:
        \[
          \sum_{j : (i^\circ, j) \in A} x_{i^\circ j} = 1
        \]

  \item \textbf{Warunek dotarcia do miasta docelowego \( j^\circ \)}:
        \[
          \sum_{i : (i, j^\circ) \in A} x_{ij^\circ} = 1
        \]

  \item \textbf{Warunek przepływu dla pozostałych miast}:
        Każde miasto \( k \in N \), poza \( i^\circ \) i \( j^\circ \), ma tyle
        samo połączeń wchodzących, ile wychodzących. W rzeczywistości będzie to
        0 lub 1 połączenie,
        bo wracanie do tego samego miasta nieporzebnie przedłuża podróż (czas i
        koszt).
        \[
          \sum_{j : (k, j) \in A} x_{kj} = \sum_{i : (i, k) \in A} x_{ik}
        \]
\end{enumerate}

\subsection{Egzemplarz}

\begin{center}
  \begin{tabular}{|c|c|c|c|}
    \hline
    \( i \) & \( j \) & \( c_{ij} \) & \( t_{ij} \) \\
    \hline
    1       & 2       & 3            & 4            \\
    1       & 3       & 4            & 9            \\
    1       & 4       & 7            & 10           \\
    1       & 5       & 8            & 12           \\
    2       & 3       & 2            & 3            \\
    3       & 4       & 4            & 6            \\
    3       & 5       & 2            & 2            \\
    3       & 10      & 6            & 11           \\
    4       & 5       & 1            & 1            \\
    4       & 7       & 3            & 5            \\
    5       & 6       & 5            & 6            \\
    5       & 7       & 3            & 3            \\
    5       & 10      & 5            & 8            \\
    6       & 1       & 5            & 8            \\
    6       & 7       & 2            & 2            \\
    6       & 10      & 7            & 11           \\
    7       & 3       & 4            & 6            \\
    7       & 8       & 3            & 5            \\
    7       & 9       & 1            & 1            \\
    8       & 9       & 1            & 2            \\
    9       & 10      & 2            & 2            \\
    \hline
  \end{tabular}
\end{center}

\begin{center}
  \begin{tabular}{|c|c|}
    \hline
    \( N \)       & \{0, 1, \dots, 10\} \\
    \( i^\circ \) & 1                   \\
    \( j^\circ \) & 10                  \\
    \( T \)       & 15                  \\
    \hline
  \end{tabular}
\end{center}

\subsection{Rozwiązanie}

\begin{center}
  \begin{tabular}{|c|c|c|c|}
    \hline
    \( i \) & \( j \) & \( c_{ij} \) & \( t_{ij} \) \\
    \hline
    1       & 2       & 3            & 4            \\
    2       & 3       & 2            & 3            \\
    3       & 5       & 2            & 2            \\
    5       & 7       & 3            & 3            \\
    7       & 9       & 1            & 1            \\
    9       & 10      & 2            & 2            \\
    \hline
  \end{tabular}
\end{center}

\begin{center}
  \begin{tabular}{|c|c|}
    \hline
    Czas  & 15 \\
    Koszt & 13 \\
    \hline
  \end{tabular}
\end{center}

\subsection{Egzemplarz 2}

\begin{center}
  \begin{tabular}{|c|c|c|c|}
    \hline
    \( i \) & \( j \) & \( c_{ij} \) & \( t_{ij} \) \\
    \hline
    1       & 2       & 5            & 4            \\
    2       & 3       & 4            & 2            \\
    3       & 10      & 7            & 6            \\
    1       & 4       & 6            & 5            \\
    4       & 10      & 4            & 8            \\
    \hline
  \end{tabular}
\end{center}

\begin{center}
  \begin{tabular}{|c|c|}
    \hline
    \( N \)       & \{0, 1, \dots, 10\} \\
    \( i^\circ \) & 1                   \\
    \( j^\circ \) & 10                  \\
    \( T \)       & 12                  \\
    \hline
  \end{tabular}
\end{center}

\subsection{Rozwiązanie 2}

\begin{center}
  \begin{tabular}{|c|c|c|c|}
    \hline
    \( i \) & \( j \) & \( c_{ij} \) & \( t_{ij} \) \\
    \hline
    1       & 2       & 5            & 4            \\
    2       & 3       & 4            & 2            \\
    3       & 10      & 7            & 6            \\
    \hline
  \end{tabular}
\end{center}

\begin{center}
  \begin{tabular}{|c|c|}
    \hline
    Czas  & 12 \\
    Koszt & 16 \\
    \hline
  \end{tabular}
\end{center}

\begin{itemize}
  \item Czy zmienne całkowitoliczbowe są konieczne? \\
        Czy po usunięciu ograniczenia na czasy przejazdu w modelu bez
        ograniczeń na całkowitoliczbowość zmiennych decyzyjnych i rozwiązaniu
        problemu
        otrzymane połączenie zawsze jest akceptowalnym rozwiązaniem? \\
        Ograniczenie to nie jest potrzebne. Po jego zdjęciu otrzymujemy takie
        samo rozwiązanie.
        Pomijając ograniczenie czasu, problem możemy zaprezentować jako $ M
          x = b $
        gdzie M to macierz ograniczeń (macierz incydencji grafu),
        x to wektor przepływu na krawędziach i b to wektor prawych stron.
        $ \\ b(i^\circ) = 1 \\ b(j^\circ) = -1 \\ b(a) = 0, a \neq i^\circ,
          j^\circ \\ $
        Ponieważ M jest całkowicie unimodularna i b jest wektorem liczb
        całkowitych,
        każde podstawowe rozwiązanie jest całkowitoliczbowe.
        Dodanie ponownie ograniczenia czasu zburza jednak całkowitoliczbowość.
\end{itemize}

\begin{center}
  \begin{tabular}{|c|c|c|c|}
    \hline
    \( i \) & \( j \) & \( c_{ij} \) & \( t_{ij} \) \\
    \hline
    1       & 2       & 1            & 3            \\
    2       & 10      & 1            & 3            \\
    1       & 10      & 1            & 9            \\
    \hline
  \end{tabular}
\end{center}

\begin{center}
  \begin{tabular}{|c|c|}
    \hline
    \( N \)       & \{0, 1, \dots, 10\} \\
    \( i^\circ \) & 1                   \\
    \( j^\circ \) & 10                  \\
    \( T \)       & 8                   \\
    \hline
  \end{tabular}
\end{center}

Wtedy rozwiązanie optymalne nie jest całkowitoliczbowe.
\[ Koszt = 1 \frac{1}{3} \]
Przesyłamy $\frac{1}{3}$ ścieżką długości 2 i $\frac{2}{3}$ ścieżką długości 1.

\section{Zadanie 5}

Model ten rozwiązuje problem przydziału radiowozów do trzech dzielnic w
miasteczku, z uwzględnieniem minimalnych i maksymalnych liczby radiowozów dla
każdej zmiany oraz minimalnych wymagań dla dzielnic i zmian.

\subsection{Parametry}
\begin{itemize}
  \item \( m \): liczba dzielnic,
  \item \( n \): liczba zmian,
  \item \( Min_{ij} \): minimalna liczba radiowozów dla danej dzielnicy i
        zmiany,
  \item \( Max_{ij} \): maksymalna liczba radiowozów dla danej dzielnicy i
        zmiany,
  \item \( m_j \): minimalna liczba radiowozów dla danej zmiany,
  \item \( d_i \): minimalna liczba radiowozów dla danej dzielnicy.
\end{itemize}

\subsection{Zmienne decyzyjne}
Definiujemy zmienne decyzyjne \( x_{ij} \), które reprezentują liczbę
radiowozów przydzielonych do dzielnicy \( i \) w zmianie \( j \):
\[
  x_{ij} \geq 0
\]

\subsection{Funkcja celu}
Celem jest minimalizacja całkowitej liczby radiowozów. Funkcja celu wyraża się
następująco:
\[
  \min \sum_{i=1}^{m} \sum_{j=1}^{n} x_{ij}
\]

\subsection{Ograniczenia}
\begin{enumerate}
  \item \textbf{Ograniczenia minimalnej liczby radiowozów dla każdej zmiany}:
        Dla każdej zmiany \( j \) muszą być dostępne określone liczby
        radiowozów, oznaczone jako \( m_j \). Ograniczenia te mają postać:
        \[
          \sum_{i=1}^{m} x_{ij} \geq m_j
        \]

  \item \textbf{Ograniczenia minimalnej liczby radiowozów dla dzielnic}:
        Każda dzielnica \( i \) powinna mieć przypisaną określoną minimalną
        liczbę radiowozów, oznaczoną jako \( d_i \). Ograniczenie dla każdej
        dzielnicy
        ma postać:
        \[
          \sum_{j=1}^{n} x_{ij} \geq d_i
        \]

  \item \textbf{Ograniczenia minimalnej i maksymalnej liczby radiowozów dla
          każdej zmiany i
          dzielnicy}:
        Dla każdej dzielnicy \( i \) i każdej zmiany \( j \) nie może być
        więcej radiowozów niż maksymalna liczba i mniej niż minimalna liczba:
        \[
          Min_{ij} \leq x_{ij} \leq Max_{ij}
        \]
\end{enumerate}

\subsection{Egzemplarz}

\begin{center}
  \begin{tabular}{|c|c|c|c|}
    \hline
    \( Min_{ij}, Max_{ij} \) & 1    & 2     & 3     \\
    \hline
    1                        & 2, 3 & 4, 5  & 3, 7  \\
    2                        & 3, 5 & 6, 7  & 5, 10 \\
    3                        & 5, 8 & 7, 12 & 8, 10 \\
    \hline
  \end{tabular}
\end{center}

\begin{center}
  \begin{tabular}{|c|c|c|c|}
    \hline
              & 1  & 2  & 3  \\
    \hline
    \( m_j \) & 10 & 20 & 18 \\
    \( d_i \) & 10 & 14 & 13 \\
    \hline
  \end{tabular}
\end{center}

\subsection{Rozwiązanie}

\begin{center}
  \begin{tabular}{|c|c|c|c|}
    \hline
    \( x_{ij} \) & 1 & 2 & 3 \\
    \hline
    1            & 2 & 5 & 5 \\
    2            & 3 & 7 & 5 \\
    3            & 5 & 8 & 8 \\
    \hline
  \end{tabular}
\end{center}

Liczba radiowozów = 48

\begin{itemize}
  \item Czy zmienne całkowitoliczbowe są konieczne? \\
        Ograniczenie to nie jest potrzebne, ponieważ można go zredukować
        do problemu min cost flow (w szczególności problemu cyrkulacji).
        Grafem jest źródło (komenda) połączona z dzielnicami (min na
        krawędziach), a one następnie
        połączone są ze zmianami (min i max na krawędziach), a te wierzchołki
        wracają do komendy (min na krawędziach).
\end{itemize}

\section{Zadanie 6}

Model ten rozwiązuje problem rozmieszczenia kamer w terenie składowym, w którym
kontenery z cennym ładunkiem muszą być monitorowane. Teren podzielony jest na
siatkę o wymiarach \( m \times n \), a kamery muszą być umieszczone w taki
sposób, aby każdy kontener był monitorowany przez co najmniej jedną kamerę.
Celem jest minimalizacja liczby użytych kamer.

\subsection{Parametry}
\begin{itemize}
  \item \( C_{ij} \in \{0, 1\} \): rozmieszczenie kontenerów,
  \item \( k \): zasięg kamery,
  \item \( m \): liczba wierszy w siatce,
  \item \( n \): liczba kolumn w siatce.
\end{itemize}

\subsection{Zmienne decyzyjne}
Definiujemy zmienne decyzyjne \( x_{ij} \), które reprezentują, czy kamera jest
umieszczona w kwadracie \( (i, j) \):
\[
  x_{ij} \geq 0
\]
oraz \( x_{ij} \) jest liczbą całkowitą (0 lub 1).

\subsection{Funkcja celu}
Celem jest minimalizacja liczby kamer:
\[
  \min \sum_{i=1}^{m} \sum_{j=1}^{n} x_{ij}
\]

\subsection{Ograniczenia}
\begin{enumerate}
  \item \textbf{Ograniczenie dotyczące kontenerów}:
        Dla każdego kontenera \( (i, j) \) musi być spełniony warunek, że suma
        kamer w zasięgu wynosi co najmniej 1:
        \[
          \sum_{k=\max(i-k, 1)}^{\min(i+k, m)} x_{kj} + \sum_{l=\max(j-k,
            1)}^{\min(j+k, n)}
          x_{il} \geq 1
        \]

  \item \textbf{Ograniczenie dotyczące braku kamer w kontenerach}:
        Nie można umieszczać kamer w kwadratach zajmowanych przez kontenery:
        \[
          x_{ij} = 0 \quad \text{jeżeli} \quad C_{ij} = 1
        \]
\end{enumerate}

\subsection{Egzemplarz}

\begin{center}
  \begin{tabular}{|c|c|c|c|c|}
    \hline
    0 & 0 & 1 & 0 & 0 \\ \hline
    1 & 0 & 0 & 1 & 0 \\ \hline
    0 & 1 & 0 & 0 & 0 \\ \hline
    0 & 0 & 0 & 0 & 0 \\ \hline
    0 & 1 & 1 & 0 & 1 \\ \hline
    1 & 0 & 1 & 0 & 0 \\
    \hline
  \end{tabular}
  \captionof{table}{$C_{ij}$}
\end{center}

\subsection{Rozwiązanie 1}

k = 2

\begin{center}
  \begin{tabular}{|c|c|c|c|c|}
    \hline
    0  & 1  & -1 & 0  & 0  \\ \hline
    -1 & 0  & 1  & -1 & 0  \\ \hline
    0  & -1 & 0  & 0  & 0  \\ \hline
    0  & 0  & 0  & 0  & 0  \\ \hline
    1  & -1 & -1 & 0  & -1 \\ \hline
    -1 & 0  & -1 & 0  & 1  \\
    \hline
  \end{tabular}
  \captionof{table}{$ x_{ij}, C_{ij} $ oznaczone -1}
\end{center}

Liczba kamer = 4

\subsection{Rozwiązanie 2}

k = 3

\begin{center}
  \begin{tabular}{|c|c|c|c|c|}
    \hline
    0  & 0  & -1 & 0  & 0  \\ \hline
    -1 & 0  & 0  & -1 & 0  \\ \hline
    0  & -1 & 1  & 0  & 0  \\ \hline
    0  & 0  & 0  & 0  & 0  \\ \hline
    1  & -1 & -1 & 1  & -1 \\ \hline
    -1 & 0  & -1 & 0  & 0  \\
    \hline
  \end{tabular}
  \captionof{table}{$ x_{ij}, C_{ij} $ oznaczone -1}
\end{center}

Liczba kamer = 3

\end{document}
