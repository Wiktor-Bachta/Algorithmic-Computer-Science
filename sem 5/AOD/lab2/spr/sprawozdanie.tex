\documentclass[12pt, letterpaper]{article}
\usepackage[utf8]{inputenc}
\usepackage[T1]{fontenc}
\usepackage{amsmath}
\title{AOD - sprawozdanie nr 2}
\author{Wiktor Bachta}
\date{Listopad 2024}

\begin{document}

\maketitle

\section{Zadanie 1}

Model ten rozwiązuje problem minimalizacji kosztów zakupu paliwa dla
przedsiębiorstwa lotniczego, które musi dostarczyć określoną ilość paliwa na
lotniska od różnych dostawców. Model oparty jest na programowaniu
liniowym i wykorzystuje bibliotekę JuMP do optymalizacji.

\subsection*{Zmienne decyzyjne}
Definiujemy zmienne decyzyjne \( x_{ij} \), które reprezentują ilość paliwa
dostarczanego przez firmę \( i \) na lotnisko \( j \), gdzie:
\[
  x_{ij} \geq 0
\]
oraz \( x_{ij} \) jest liczbą całkowitą.

\subsection*{Funkcja celu}
Celem jest minimalizacja całkowitego kosztu dostarczenia paliwa. Koszt
dostarczenia jednego galonu paliwa przez dostawcę \( i \) na lotnisko \( j \)
jest zapisany w macierzy kosztów \( c_{ij} \). Funkcja celu wyraża się
następująco:
\[
  \min \sum_{i=1}^m \sum_{j=1}^n c_{ij} x_{ij}
\]
gdzie:
\begin{itemize}
  \item $m$: liczba dostawców,
  \item $n$: liczba lotnisk.
\end{itemize}

\subsection*{Ograniczenia}
\begin{enumerate}
  \item \textbf{Ograniczenia zapotrzebowania na paliwo na każdym lotnisku}:
        Każde lotnisko \( j \) wymaga określonej ilości paliwa, oznaczonej jako
        \( d_j
        \). Dla każdego lotniska \( j \) formułujemy ograniczenie:
        \[
          \sum_{i=1}^m x_{ij} = d_j
        \]

  \item \textbf{Ograniczenia produkcyjne dostawców}: Każdy dostawca \( i \)
        ma ograniczoną ilość paliwa, którą może dostarczyć (zdefiniowaną jako
        \( p_i
        \)). Sformułowanie tych ograniczeń dla każdego dostawcy \( i \) wygląda
        następująco:
        \[
          \sum_{j=1}^n x_{ij} \leq p_i
        \]
\end{enumerate}

\section{Zadanie 2}

Model ten rozwiązuje problem optymalizacji produkcji zakładu, który może
wytwarzać cztery różne wyroby w ograniczonych ilościach, uwzględniając czas
przetwarzania na trzech różnych maszynach oraz popyt rynkowy. Celem modelu jest
maksymalizacja zysku.

\subsection*{Zmienne decyzyjne}
Definiujemy zmienne decyzyjne \( x_{ij} \), które reprezentują ilość kilogramów
wyrobu \( P_i \), produkowanego na maszynie \( M_j \), gdzie:
\[
  x_{ij} \geq 0
\]
oraz \( x_{ij} \) jest liczbą całkowitą.

\subsection*{Funkcja celu}
Celem jest maksymalizacja zysku, który można wyrazić jako różnicę między
przychodem ze sprzedaży wyrobów a kosztami zmiennymi, które obejmują koszty
materiałów oraz koszty pracy maszyn. Funkcja celu przyjmuje postać:
\[
  \max \sum_{i=1}^n \sum_{j=1}^m \left( \left( p_i - m_i \right) \cdot	x_{ij}
  - k_j \cdot t_{ij} \cdot x_{ij} \right)
\]

gdzie:
\begin{itemize}
  \item \( p_i \): cena sprzedaży wyrobu \( P_i \),
  \item \( m_i \): koszt materiałowy na kilogram wyrobu \( P_i \),
  \item \( k_j \): koszt pracy maszyny \( M_j \) za minutę,
  \item \( t_{ij} \): czas przetwarzania jednego kilograma wyrobu \( P_i \) na
        maszynie \( M_j \).
\end{itemize}

\subsection*{Ograniczenia}
\begin{enumerate}
  \item \textbf{Ograniczenia dostępności czasu pracy maszyn}: Każda maszyna \(
        M_j \) jest dostępna tylko przez określoną liczbę minut tygodniowo,
        oznaczoną
        jako \( c_j \). Dla każdej maszyny \( M_j \) formułujemy ograniczenie:
        \[
          \sum_{i=1}^n t_{ij} \cdot x_{ij} \leq c_j
        \]

  \item \textbf{Ograniczenia popytu rynkowego}: Każdy produkt \( P_i \) ma
        maksymalny popyt rynkowy, oznaczony jako \( d_i \). Dla każdego wyrobu
        \( P_i
        \) formułujemy ograniczenie:
        \[
          \sum_{j=1}^m x_{ij} \leq d_i
        \]
\end{enumerate}

\section{Zadanie 3}

Model ten rozwiązuje problem optymalizacji produkcji i magazynowania towarów w
firmie, która posiada ograniczoną zdolność produkcyjną w trybie podstawowym
oraz dodatkowym, przy czym produkcja dodatkowa wiąże się z wyższym kosztem
jednostkowym. Firma musi spełnić zapotrzebowanie w każdym z okresów przy
minimalizacji kosztów produkcji i magazynowania.

\subsection*{Zmienne decyzyjne}
Wprowadzamy następujące zmienne decyzyjne:
\begin{itemize}
  \item \( x_j \): liczba jednostek wyprodukowanych w trybie podstawowym w
        okresie \( j \), gdzie \( x_j \geq 0 \),
  \item \( y_j \): liczba jednostek wyprodukowanych w trybie dodatkowym w
        okresie \( j \), gdzie \( y_j \geq 0 \),
  \item \( s_j \): liczba jednostek towaru przechowywanych w magazynie na
        koniec okresu \( j \), gdzie \( s_j \geq 0 \).
\end{itemize}

\subsection*{Funkcja celu}
Celem jest minimalizacja łącznych kosztów produkcji i magazynowania towaru.
Funkcja celu przyjmuje postać:
\[
  \min \sum_{j=1}^K \left( c_j x_j + o_j y_j + h s_j \right)
\]
gdzie:
\begin{itemize}
  \item \( c_j \): koszt produkcji jednej jednostki w trybie podstawowym w
        okresie \( j \),
  \item \( o_j \): koszt produkcji jednej jednostki w trybie dodatkowym w
        okresie \( j \),
  \item \( h \): koszt magazynowania jednej jednostki towaru przez jeden okres.
\end{itemize}

\subsection*{Ograniczenia}
\begin{enumerate}
  \item \textbf{Ograniczenia zdolności produkcyjnych}:
        \[
          x_j \leq b_j
        \]
        \[
          y_j \leq a_j
        \]
        gdzie \( b_j \) i \( a_j \) to odpowiednio maksymalna liczba jednostek
        produkowanych w trybie podstawowym i dodatkowym w okresie \( j \).

  \item \textbf{Ograniczenia zapotrzebowania na towar}:
        Liczba jednostek towaru przechowywanych na koniec okresu \( j+1 \) musi
        zapewnić spełnienie zapotrzebowania w kolejnym okresie, zatem:
        \[
          s_{j+1} = s_j + x_j + y_j - d_j
        \]
        gdzie \( d_j \) to zapotrzebowanie w okresie \( j \).

  \item \textbf{Ograniczenia magazynowe}:
        Przechowywana liczba jednostek towaru na koniec każdego okresu nie może
        przekroczyć maksymalnej pojemności magazynu:
        \[
          s_j \leq S
        \]
        gdzie \( S \) to maksymalna liczba jednostek, które mogą być
        przechowywane.

  \item \textbf{Warunki początkowe i końcowe}:
        Na początku pierwszego okresu stan magazynu wynosi \( s_1 = s_0 \),
        gdzie \( s_0 \) to początkowa liczba jednostek towaru. Na koniec
        ostatniego
        okresu \( s_{K+1} = 0 \), co oznacza, że na koniec cyklu produkcyjnego
        magazyn
        ma zostać opróżniony.
\end{enumerate}

\section{Zadanie 4}
Problem ten dotyczy znalezienia najtańszego połączenia między dwoma miastami w
sieci połączeń, gdzie całkowity czas przejazdu nie może przekroczyć zadanego
limitu.

\subsection*{Model matematyczny}
Model ten jest reprezentowany przez skierowany graf \( G = (N, A) \), gdzie:
\begin{itemize}
  \item \( N \): zbiór miast, numerowanych od \( 1 \) do \( 10 \),
  \item \( A \): zbiór połączeń między miastami, każdy w postaci \( (i, j,
        c_{ij}, t_{ij}) \), gdzie \( c_{ij} \) to koszt przejazdu, a \( t_{ij}
        \) to
        czas przejazdu.
\end{itemize}

\subsection*{Zmienne decyzyjne}
\begin{itemize}
  \item \( x_{ij} \): zmienna decyzyjna, która przyjmuje wartość \( 1 \),
        jeśli połączenie z miasta \( i \) do miasta \( j \) jest używane w
        optymalnej
        ścieżce, i \( 0 \) w przeciwnym przypadku.
\end{itemize}

\subsection*{Funkcja celu}
Minimalizacja całkowitego kosztu przejazdu:
\[
  \min \sum_{(i, j) \in A} c_{ij} \cdot x_{ij}
\]

\subsection*{Ograniczenia}
\begin{enumerate}
  \item \textbf{Ograniczenie dotyczące istniejących połaczeń}:
        \[
          x_{ij} = 0 \quad \text{jeżeli nie ma połączenia z i do j}
        \]
  \item \textbf{Ograniczenie czasu przejazdu}:
        \[
          \sum_{(i, j) \in A} t_{ij} \cdot x_{ij} \leq T
        \]
        gdzie \( T \) to maksymalny dopuszczalny czas przejazdu.

  \item \textbf{Warunek wyjścia z miasta początkowego \( i^\circ \)}:
        \[
          \sum_{j : (i^\circ, j) \in A} x_{i^\circ j} = 1
        \]

  \item \textbf{Warunek dotarcia do miasta docelowego \( j^\circ \)}:
        \[
          \sum_{i : (i, j^\circ) \in A} x_{ij^\circ} = 1
        \]

  \item \textbf{Warunek przepływu dla pozostałych miast}:
        Każde miasto \( k \in N \), poza \( i^\circ \) i \( j^\circ \), ma tyle
        samo połączeń wchodzących, ile wychodzących:
        \[
          \sum_{j : (k, j) \in A} x_{kj} = \sum_{i : (i, k) \in A} x_{ik}
        \]
\end{enumerate}

\subsection*{Część (a) Rozwiązanie dla przykładowego problemu}
Dane dla egzemplarza problemu to:
\begin{itemize}
  \item \( N = \{1, \dots, 10\} \),
  \item \( i^\circ = 1 \), \( j^\circ = 10 \),
  \item \( T = 15 \),
  \item Krawędzie wraz z kosztami \( c_{ij} \) i czasami \( t_{ij} \):\\
        \( (1, 2, 3, 4), (1, 3, 4, 9), (1, 4, 7, 10), (1, 5, 8, 12), (2, 3, 2,
        3),
        (3, 4, 4, 6), (3, 5, 2, 2), (3, 10, 6, 11), \\
        (4, 5, 1, 1), (4, 7, 3, 5), (5, 6, 5, 6), (5, 7, 3, 3), (5, 10, 5, 8),
        (6,
        1, 5, 8), (6, 7, 2, 2), (6, 10, 7, 11), \\
        (7, 3, 4, 6), (7, 8, 3, 5), (7, 9, 1, 1), (8, 9, 1, 2), (9, 10, 2, 2)
        \).
\end{itemize}

Po zastosowaniu modelu kodem w Julia (JuMP), wyznaczamy optymalną ścieżkę
minimalizującą koszt przejazdu w zadanym czasie \( T \).

\subsection*{Część (b) Propozycja własnego egzemplarza problemu}
Tworzymy problem z co najmniej \( 10 \) miastami i krawędziami tak dobranymi,
aby najtańsza trasa spełniająca ograniczenie czasu miała co najmniej \( 3 \)
krawędzie i większy koszt niż najtańsza trasa bez ograniczeń. Przykład może
obejmować:
\begin{itemize}
  \item \( N = \{1, \dots, 10\} \),
  \item \( i^\circ = 1 \), \( j^\circ = 10 \), \( T = 12 \),
  \item Krawędzie: \( (1, 2, 5, 4), (2, 3, 4, 2), (3, 10, 7, 6), (1, 4, 6,
        5), (4, 10, 4, 4) \).
\end{itemize}
Rozwiązując model, znajdziemy ścieżkę spełniającą ograniczenia czasu oraz
drugą, najtańszą trasę bez tego ograniczenia.

\subsection*{Część (c) Czy zmienne całkowitoliczbowe są konieczne?}
Ograniczenie na całkowitoliczbowość jest potrzebne. Bez tego model nie zapewni
wyraźnego wyboru połączeń w ścieżce, co prowadziłoby do wyników w postaci
ułamkowej – czyli częściowych przejazdów między miastami. Przykładowo, dla
ścieżki \( (1 \to 2), (2 \to 10) \) przy założeniu minimalizacji kosztów bez
ograniczeń, optymalizacja mogłaby wyznaczyć połączenia z wartościami
ułamkowymi, co jest nielogiczne dla problemu tras przejazdu.

\subsection*{Część (d) Usunięcie ograniczenia czasu i wpływ na poprawność
  rozwiązania}
Jeśli usuniemy ograniczenie na czas przejazdu, otrzymana ścieżka minimalizująca
koszty niekoniecznie będzie spełniać wcześniejsze ograniczenie czasowe. Zatem,
mimo że rozwiązanie optymalizacyjne bez ograniczeń będzie miało najniższy
koszt, nie zawsze będzie dopuszczalne, jeśli wciąż zależy nam na spełnieniu
limitu czasu \( T \).

\section{Zadanie 5}

Model ten rozwiązuje problem przydziału radiowozów do trzech dzielnic w
miasteczku, z uwzględnieniem minimalnych i maksymalnych liczby radiowozów dla
każdej zmiany oraz minimalnych wymagań dla dzielnic i zmian. Model oparty jest
na programowaniu liniowym i wykorzystuje bibliotekę JuMP do optymalizacji.

\subsection*{Zmienne decyzyjne}
Definiujemy zmienne decyzyjne \( x_{ijk} \), które reprezentują liczbę
radiowozów przydzielonych do dzielnicy \( i \) w zmianie \( j \), gdzie:
\[
  x_{ijk} \geq 0
\]
oraz \( x_{ijk} \) jest liczbą całkowitą.

\subsection*{Funkcja celu}
Celem jest minimalizacja całkowitej liczby radiowozów. Funkcja celu wyraża się
następująco:
\[
  \min \sum_{i=1}^{3} \sum_{j=1}^{3} x_{ijk}
\]
gdzie:
\begin{itemize}
  \item \( 3 \): liczba dzielnic (p1, p2, p3),
  \item \( 3 \): liczba zmian.
\end{itemize}

\subsection*{Ograniczenia}
\begin{enumerate}
  \item \textbf{Ograniczenia minimalnej liczby radiowozów dla każdej zmiany}:
        Dla każdej zmiany \( j \) muszą być dostępne określone liczby
        radiowozów, oznaczone jako \( m_j \). Ograniczenia te mają postać:
        \[
          \sum_{i=1}^{3} x_{ijk} \geq m_j
        \]

  \item \textbf{Ograniczenia minimalnej liczby radiowozów dla dzielnic}:
        Każda dzielnica \( i \) powinna mieć przypisaną określoną minimalną
        liczbę radiowozów, oznaczoną jako \( d_i \). Ograniczenie dla każdej dzielnicy
        ma postać:
        \[
          \sum_{j=1}^{3} x_{ijk} \geq d_i
        \]

  \item \textbf{Ograniczenia maksymalnej liczby radiowozów dla każdej zmiany i
          dzielnicy}:
        Dla każdej dzielnicy \( i \) i każdej zmiany \( j \) nie może być
        więcej radiowozów niż maksymalna liczba, oznaczona jako \( M_{ij} \):
        \[
          x_{ijk} \leq M_{ij}
        \]
\end{enumerate}

\section{Zadanie 6}

Model ten rozwiązuje problem rozmieszczenia kamer w terenie składowym, w którym
kontenery z cennym ładunkiem muszą być monitorowane. Teren podzielony jest na
siatkę o wymiarach \( m \times n \), a kamery muszą być umieszczone w taki
sposób, aby każdy kontener był monitorowany przez co najmniej jedną kamerę.
Celem jest minimalizacja liczby użytych kamer.

\subsection*{Zmienne decyzyjne}
Definiujemy zmienne decyzyjne \( x_{ij} \), które reprezentują, czy kamera jest
umieszczona w kwadracie \( (i, j) \):
\[
  x_{ij} \geq 0
\]
oraz \( x_{ij} \) jest liczbą całkowitą (0 lub 1).

\subsection*{Funkcja celu}
Celem jest minimalizacja liczby kamer:
\[
  \min \sum_{i=1}^{m} \sum_{j=1}^{n} x_{ij}
\]
gdzie:
\begin{itemize}
  \item \( m \): liczba wierszy w siatce,
  \item \( n \): liczba kolumn w siatce.
\end{itemize}

\subsection*{Ograniczenia}
\begin{enumerate}
  \item \textbf{Ograniczenie dotyczące kontenerów}:
        Dla każdego kontenera \( (i, j) \) musi być spełniony warunek, że suma
        kamer w zasięgu wynosi co najmniej 1:
        \[
          \sum_{k=max(i-k, 1)}^{min(i+k, m)} \sum_{l=max(j-k, 1)}^{min(j+k, n)}
          x_{kl} \geq 1
        \]
        gdzie \( k \) to zasięg kamery.

  \item \textbf{Ograniczenie dotyczące braku kamer w kontenerach}:
        Nie można umieszczać kamer w kwadratach zajmowanych przez kontenery:
        \[
          x_{ij} = 0 \quad \text{jeżeli } containers_{ij} = 1
        \]
\end{enumerate}

\end{document}
